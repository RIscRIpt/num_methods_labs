\documentclass[a4paper]{article}
\usepackage[autostyle]{csquotes}
\usepackage{fix-cm}
\usepackage[T2A]{fontenc}
\usepackage[utf8]{inputenc}
\usepackage[russian]{babel}
\usepackage{datetime}
\usepackage{mathtools}
\usepackage{marginnote}
\usepackage{geometry}
\usepackage{titlesec}
\usepackage{nicefrac}
\usepackage{bm}
\usepackage{listings}
\usepackage{bashful}
\usepackage{pgffor}
\usepackage{amsmath}
\usepackage{amsthm}
\usepackage{amsfonts}
\usepackage{enumitem}
\usepackage{svg}
\usepackage{color}
\usepackage{xcolor}
\usepackage{colortbl}
\usepackage{tabularx}
\usepackage{multirow}
\usepackage{multicol}
\usepackage{makecell}
\usepackage{booktabs}
\usepackage{array}
\usepackage[normalem]{ulem}
\usepackage{environ}
\usepackage{caption}
% \usepackage[noae]{Sweave}

\geometry{
    a4paper,
    top=20mm,
    right=15mm,
    bottom=20mm,
    left=30mm,
}

\setcounter{secnumdepth}{4}

\titleformat{\paragraph}
{\normalfont\normalsize\bfseries}{\theparagraph}{1em}{}
\titlespacing*{\paragraph}
{0pt}{3.25ex plus 1ex minus .2ex}{1.5ex plus .2ex}

% \delimitershortfall-1sp

\newcommand{\includeimage}[2]{
    \begin{center}
        \includegraphics[width=\linewidth]{#1}
        \captionof{figure}{#2}
    \end{center}
}

\newcommand\abs[1]{\left|#1\right|}
\newcommand\norm[1]{\left|\left|#1\right|\right|}
\newcommand\cond[1]{\text{cond}\left(#1\right)}
\newcommand{\m}[1]{\ensuremath{\bm{#1}}}
\newcommand\eqdef{\mathrel{\stackrel{\makebox[0pt]{\mbox{\normalfont\tiny def}}}{=}}}

\newcommand\numberthis{\addtocounter{equation}{1}\tag{\theequation}}

\newenvironment{where}[1][где:]
    {#1 \begin{tabular}[t]{>{}<{} @{${}{}$}l}}
    {\end{tabular}\\[\belowdisplayskip]}

\newcolumntype{\l}{@{}l@{}}
\newcolumntype{\c}{@{}c@{}}
\newcolumntype{\r}{@{}r@{}}

\newcolumntype{L}{>{$}l<{$}}
\newcolumntype{C}{>{$}c<{$}}
\newcolumntype{R}{>{$}r<{$}}

\newcolumntype{\L}{@{}L@{}}
\newcolumntype{\C}{@{}C@{}}
\newcolumntype{\R}{@{}R@{}}

\newcolumntype{\ML}{>{$\displaystyle}l<{$}}
\newcolumntype{\MC}{>{$\displaystyle}c<{$}}
\newcolumntype{\MR}{>{$\displaystyle}r<{$}}

\newcolumntype{\NL}{@{}\ML @{}}
\newcolumntype{\NC}{@{}\MC @{}}
\newcolumntype{\NR}{@{}\MR @{}}

\lstset{%
    basicstyle=\ttfamily,
    keywordstyle=\color{blue}\ttfamily,
    stringstyle=\color{red}\ttfamily,
    commentstyle=\color{green}\ttfamily,
    morecomment=[l][\color{magenta}]{\#},
    showstringspaces=false,
}

\DeclareTextFontCommand{\emph}{\boldmath\bfseries}

\begin{document}

% \input{main-concordance}
\begin{titlepage}
    \begin{center}
        \setlength{\parindent}{0cm}
        \fontsize{16pt}{16pt}\selectfont

        \textbf{ИНСТИТУТ ТРАНСПОРТА И СВЯЗИ}

        \rule{\textwidth}{1pt}

        \vspace*{2.0cm}

        \includegraphics[width=5cm]{../../tsi_logo}

        \vspace*{2.0cm}

        ФАКУЛЬТЕТ КОМПЬЮТЕРНЫХ НАУК И ТЕЛЕКОММУНИКАЦИЙ

        \vspace*{2.0cm}

        Лабораторная работа №2 \\
        По дисциплине \\
        <<Численные методы и прикладное программирование>>

        \vspace*{2.0cm}
        \fontsize{14pt}{14pt}\selectfont
        Тема: \\
        <<Методы приближения функции. \\ Интерполяция и аппроксимация>>

        \vfill
        \fontsize{12pt}{12pt}\selectfont

        \begin{flushright}
            Работу выполнили: \\
            \vspace*{0.25cm}
            {\itshape Дзенис Ричард} \\
            {\itshape Кобелев Денис} \\
            {\itshape Якушин Владислав} \\
        \end{flushright}

        \vfill

        Рига \\ 2017 г.
    \end{center}
\end{titlepage}

\tableofcontents

\pagebreak

\section{Формулировка задания}
\begin{itemize}
    \item Реализовать программным путём аппроксимацию функции по заданным точкам методом наименьших квадратов (МНК);
    \item Реализовать программным путём интерполяцию функции по заданным точкам с помощью кубических сплайнов;
    \item Графически сравнить получившиеся результаты;
    \item Сделать выводы по реализованным методам.
\end{itemize}

\begin{equation}
    y = \sin(5x) \cdot e^x
\end{equation}
\begin{where}
    $x \in [-2; +2]; \quad \Delta x = 0.5$
\end{where}

\begin{center}
    \begin{tabular}{\MC|\MC|\MC|\MC|\MC|\MC|\MC|\MC}
        x & -3.2 & -2.1 & 0.4 & 0.7 & 2 & 2.5 & 2.777 \\
        \hline
        y & 10   & -2   & 0   & -7  & 7 & 0   & 0
    \end{tabular}
\end{center}

\section{Аппроксимация методом наименьших квадратов}

Аппроксимацию методом МНК можно разбить на 2 этапа:
\begin{enumerate}
    \item Построение матрицы коэффициентов СЛАУ согласно универсальной СЛАУ МНК;
    \item Решение СЛАУ для нахождения аппроксимирующего полинома.
\end{enumerate}

В силу того, что универсальная СЛАУ для МНК является симметричной, то для экономии памяти можно сохранять только верную или нижнюю треугольную матрицу.

\subsection{Листинги}
\subsubsection{Листинг построения нижней треугольной матрицы коэффициентов для универсальной СЛАУ МНК}
\lstinputlisting[language=c++, firstline=18, lastline=46]{../least_squares.cpp}

Для того чтобы воспользоваться уже разработанной программой решения СЛАУ методом исключения Гаусса (в первой лабораторной работе), необходимо вывести всю матрицу целиком, с вектором правой части.

\subsubsection{Листинг вывода СЛАУ}
\lstinputlisting[language=c++, firstline=60, lastline=70]{../least_squares.cpp}

Далее СЛАУ решается с помощью программы разработанной в ходе первой лабораторной работы.

\subsection{Результаты работы метода}
\includeimage{../plots/task1ols3.png}{График задания №1}

\includeimage{../plots/task2ols3.png}{График задания №2}

\section{Кубическая сплайн интерполяция}
\subsection{Листинг}
\subsubsection{Листинг функции решения СЛАУ с 3ёх диагональными матрицами коэффициентов}
\lstinputlisting[language=c++, firstline=42, lastline=60]{../spline.cpp}

\subsubsection{Листинг функции нахождения коэффициентов сплайнов}
\lstinputlisting[language=c++, firstline=62, lastline=93]{../spline.cpp}

\subsection{Результаты работы метода}
\includeimage{../plots/task1spline.png}{График сплайн интерполяции для задания №1}

\includeimage{../plots/task2spline.png}{График сплайн интерполяции для задания №2}

\section{Выводы}

\end{document}

