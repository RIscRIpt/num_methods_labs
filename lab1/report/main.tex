\documentclass[a4paper]{article}
\usepackage[autostyle]{csquotes}
\usepackage{fix-cm}
\usepackage[T2A]{fontenc}
\usepackage[utf8]{inputenc}
\usepackage[russian]{babel}
\usepackage{datetime}
\usepackage{mathtools}
\usepackage{marginnote}
\usepackage{geometry}
\usepackage{titlesec}
\usepackage{nicefrac}
\usepackage{bm}
\usepackage{listings}
\usepackage{bashful}
\usepackage{pgffor}
\usepackage{amsmath}
\usepackage{amsthm}
\usepackage{amsfonts}
\usepackage{enumitem}
\usepackage{svg}
\usepackage{color}
\usepackage{xcolor}
\usepackage{colortbl}
\usepackage{tabularx}
\usepackage{multirow}
\usepackage{multicol}
\usepackage{makecell}
\usepackage{booktabs}
\usepackage{array}
\usepackage[normalem]{ulem}
\usepackage{environ}
% \usepackage[noae]{Sweave}

\geometry{
    a4paper,
    top=20mm,
    right=15mm,
    bottom=20mm,
    left=30mm,
}

\setcounter{secnumdepth}{4}

\titleformat{\paragraph}
{\normalfont\normalsize\bfseries}{\theparagraph}{1em}{}
\titlespacing*{\paragraph}
{0pt}{3.25ex plus 1ex minus .2ex}{1.5ex plus .2ex}

\delimitershortfall-1sp

\newcommand\abs[1]{\left|#1\right|}
\newcommand\norm[1]{\left|\left|#1\right|\right|}
\newcommand\cond[1]{\text{cond}\left(#1\right)}
\newcommand{\m}[1]{\ensuremath{\bm{#1}}}
\newcommand\eqdef{\mathrel{\stackrel{\makebox[0pt]{\mbox{\normalfont\tiny def}}}{=}}}

\newcommand\numberthis{\addtocounter{equation}{1}\tag{\theequation}}

\newenvironment{where}[1][где:]
    {#1 \begin{tabular}[t]{>{}<{} @{${}{}$}l}}
    {\end{tabular}\\[\belowdisplayskip]}

\lstset{%
    basicstyle=\ttfamily,
    keywordstyle=\color{blue}\ttfamily,
    stringstyle=\color{red}\ttfamily,
    commentstyle=\color{green}\ttfamily,
    morecomment=[l][\color{magenta}]{\#},
    showstringspaces=false,
}

\DeclareTextFontCommand{\emph}{\boldmath\bfseries}

\begin{document}

% \input{main-concordance}

\begin{titlepage}
    \begin{center}
        \setlength{\parindent}{0cm}
        \fontsize{16pt}{16pt}\selectfont

        \textbf{ИНСТИТУТ ТРАНСПОРТА И СВЯЗИ}

        \rule{\textwidth}{1pt}

        \vspace*{2.0cm}

        \includegraphics[width=5cm]{tsi_logo}

        \vspace*{2.0cm}

        ФАКУЛЬТЕТ КОМПЬЮТЕРНЫХ НАУК И ТЕЛЕКОММУНИКАЦИЙ

        \vspace*{2.0cm}

        Лабораторная работа №1 \\
        По дисциплине \\
        <<Численные методы и прикладное программирование>>

        \vspace*{2.0cm}
        \fontsize{14pt}{14pt}\selectfont
        Тема: \\
        <<Методы решения системы линейных уравнений. \\ Число обусловленности матрицы>>

        \vfill
        \fontsize{12pt}{12pt}\selectfont

        \begin{flushright}
            Работу выполнили: \\
            \vspace*{0.25cm}
            {\itshape Дзенис Ричард} \\
            {\itshape Кобелев Денис} \\
            {\itshape Якушин Владислав} \\
        \end{flushright}

        \vfill
        
        Рига \\ 2017 г.
    \end{center}
\end{titlepage}

\section{Формулировка задания}
\begin{itemize}
    \item Реализовать программным путём метод исключения Гаусса и итерационный метод Гаусса-Зейделя. 
    \item Результат работы программы проверить с помощью предоставленных примеров.
    \item Ручным или программным путём рассчитать число обусловленности матриц для примеров \eqref{slau:3} и \eqref{slau:5}.
    \item Для расчёта обусловленности выбрать Манхэттенскую или Евклидову норму.
    \item Составить отчёт с результатами вычислений и выводами, содержащими сравнение двух реализованных методов, а так же объяснить значения полученный при вычислении числа обусловленности матриц.
\end{itemize}

\subsection{Примеры}

\begin{equation}
    \begin{cases*}
           x_1  -2 x_2 + x_3 = 2 \\
         2 x_1  -5 x_2 - x_3 = -1 \\
        -7 x_1  +  x_3       = -2
    \end{cases*} \label{slau:1}
\end{equation}

\begin{equation}
    \begin{cases}
         5 x_1 -5 x_2 -3 x_3 +4 x_4 = -11 \\
           x_1 -4 x_2 +6 x_3 -4 x_4 = -10 \\
        -2 x_1 -5 x_2 +4 x_3 -5 x_4 = -12 \\
        -3 x_1 -3 x_2 +5 x_3 -5 x_4 = 8
    \end{cases} \label{slau:2}
\end{equation}

\begin{equation}
    \begin{cases*}
        2 x_1 -   x_2 -   x_3 = 5 \\
          x_1 + 3 x_2 - 2 x_3 = 7 \\
          x_1 + 2 x_2 + 3 x_3 = 10
    \end{cases*} \label{slau:3}
\end{equation}

\begin{equation}
    \begin{cases}
         8 x_1 + 5 x_2  + 3  x_3 = 30 \\
        -2 x_1 + 8 x_2  +    x_3 = 15 \\
           x_1 + 3 x_2  - 10 x_3 = 42
    \end{cases} \label{slau:4}
\end{equation}

\begin{equation}
    \begin{cases}
        0.78  x_1 + 0.563 x_2 = 0.217 \\
        0.913 x_1 + 0.659 x_2 = 0.254
    \end{cases} \label{slau:5}
\end{equation}

\break

\section{Метод исключения Гаусса с ведущим элементом}

\subsection{Листинг}
\lstinputlisting[language=c++, firstline=5, lastline=49]{../gauss.cpp}

\subsection{Результаты работы алгоритма}

\subsubsection{Результат работы алгоритма на примере \eqref{slau:1}}
\bash[stdout]
../solve_problem gauss 1
\END

\subsubsection{Результат работы алгоритма на примере \eqref{slau:2}}
\bash[stdout]
../solve_problem gauss 2
\END

\subsubsection{Результат работы алгоритма на примере \eqref{slau:5}}
\bash[stdout]
../solve_problem gauss 5
\END

\section{Метод Гаусса-Зейделя}

\subsection{Листинг}
\lstinputlisting[language=c++, firstline=5, lastline=27]{../gauss-seidel.cpp}

\subsection{Результаты работы алгоритма с точностью $10^{-3}$}

\subsubsection{Результат работы алгоритма на примере \eqref{slau:1}}
\bash[stdout]
../solve_problem gauss-seidel 1 0.001
\END
\newline

В данном примере не выполняется условие сходимости для итерационного метода, что и было успешно обнаружено программой.

\subsubsection{Результат работы алгоритма на примере \eqref{slau:3}}
\bash[stdout]
../solve_problem gauss-seidel 3 0.001
\END

\subsubsection{Результат работы алгоритма на примере \eqref{slau:4}}
\bash[stdout]
../solve_problem gauss-seidel 4 0.001
\END

\section{Экспериментальное определение числа обусловленности матрицы}

\subsection{Часть листинга для нахождения числа обусловленности}
\lstinputlisting[language=python, firstline=35, lastline=49]{../find_cond.py}

\subsection{Полученные результаты}

При нахождении числа обусловленности экспериментальным методом, для расчётов неизвестных ($\m{X}$) использовался метод исключения Гаусса, для получения более высокой точности. А также, из-за того, что итерационным методом некоторые СЛАУ не возможно решить.

\paragraph{Пример \eqref{slau:1}}
\bash[stdout]
../find_cond.py 1 gauss
\END

\paragraph{Пример \eqref{slau:2}}
\bash[stdout]
../find_cond.py 2 gauss
\END

\paragraph{Пример \eqref{slau:5}}
\bash[stdout]
../find_cond.py 5 gauss
\END

\section{Выводы}
В ходе выполнения данной лабораторной работы были реализованы два алгоритма нахождения решения СЛАУ на языке C++:
\begin{itemize}
    \item Прямой метод: <<метод исключения Гаусса>>;
    \item Итерационный метод: <<метод Гаусса-Зейделя>>.
\end{itemize}
А также были вычислены числа обусловленности для заданных примеров экспериментальным путём.

\bigskip

При сравнении двух реализованных методов можно заметить, что метод Гаусса-Зейделя не предназначен для решения всевозможных СЛАУ.
При проверки работы алгоритмов на предоставленных примерах, в 1-ом и 2-ом примере метод Гаусса-Зейделя не сходиться. Примеры 3 и 4 имели довольно схожие результаты.
5-ый пример при низкой точности ($\varepsilon \approx 10^{-6}$), результаты, полученные методом Гаусса-Зейделя довольно сильно расходиться с методом исключения Гаусса. 

\bigskip

Разница времени выполнения обоих методов для данных (малых) СЛАУ была довольно незначительной, поэтому, для сравнения времени выполнения, было принято решение создать СЛАУ размером $1500 \times 1500$, и произвести замеры на ней.

Каждый алгоритм был запущен $10$ раз на одной и той-же матрице $1500 \times 1500$, в результате получено:
\begin{center}
    \begin{tabular}{r|cc}
        Метод & Среднее время выполнения (с) & Точность \\
        \hline
        Метод исключения Гаусса & $5.357956087$ & $\pm 0.33\%$ \\
        Метод Гаусса-Зейделя    & $0.655645297$ & $\pm 0.73\%$ \\
    \end{tabular}
\end{center}

Полученные результаты были ожидаемы, т.к. метод исключения Гаусса имеет алгоритмическую сложность $O(n^3)$, тогда как метод Гаусса-Зейделя: $O(mn^2)$, где $m$ -- количество итераций, которое необходимо для достижения требуемой точности.
\begin{equation*}
    O(n^3) > O(mn^2) \qquad \text{если} \; n \gg m
\end{equation*}

Таким образом при не больших СЛАУ (когда $n \approx m$) метод исключения Гаусса не медленнее метода Гаусса-Зейделя. Однако при увеличении размера СЛАУ, когда $n \gg m$, первый метод начинает работать значительно медленнее второго.

\bigskip

В ходе вычисления числа обусловленности Матриц была использована Евклидова норма. За исключением 5-ого примера, все числа обусловленности получились до 10, что означает хорошую обусловленность.
Число обусловленности матрицы из 5-ого уравнения составила 227109, что явно свидетельствует о плохой обусловленности, так как значение выше 1000. Скорее всего, данный факт и является причиной расхождения результатов метода Гаусса-Зейделя и метода исключений Гаусса при низкой точности.

\end{document}

