\documentclass[a4paper]{article}
\usepackage[autostyle]{csquotes}
\usepackage{fix-cm}
\usepackage[T2A]{fontenc}
\usepackage[utf8]{inputenc}
\usepackage[russian]{babel}
\usepackage{datetime}
\usepackage{mathtools}
\usepackage{marginnote}
\usepackage{geometry}
\usepackage{titlesec}
\usepackage{nicefrac}
\usepackage{bm}
\usepackage{listings}
\usepackage{bashful}
\usepackage{pgffor}
\usepackage{amsmath}
\usepackage{amsthm}
\usepackage{amsfonts}
\usepackage{enumitem}
\usepackage{svg}
\usepackage{color}
\usepackage{xcolor}
\usepackage{colortbl}
\usepackage{tabularx}
\usepackage{multirow}
\usepackage{multicol}
\usepackage{makecell}
\usepackage{booktabs}
\usepackage{array}
\usepackage[normalem]{ulem}
\usepackage{environ}
% \usepackage[noae]{Sweave}

\geometry{
    a4paper,
    top=20mm,
    right=15mm,
    bottom=20mm,
    left=30mm,
}

\setcounter{secnumdepth}{4}

\titleformat{\paragraph}
{\normalfont\normalsize\bfseries}{\theparagraph}{1em}{}
\titlespacing*{\paragraph}
{0pt}{3.25ex plus 1ex minus .2ex}{1.5ex plus .2ex}

\delimitershortfall-1sp

\newcommand\abs[1]{\left|#1\right|}
\newcommand\norm[1]{\left|\left|#1\right|\right|}
\newcommand\cond[1]{\text{cond}\left(#1\right)}
\newcommand{\m}[1]{\ensuremath{\bm{#1}}}
\newcommand\eqdef{\mathrel{\stackrel{\makebox[0pt]{\mbox{\normalfont\tiny def}}}{=}}}

\newcommand\numberthis{\addtocounter{equation}{1}\tag{\theequation}}

\newenvironment{where}[1][где:]
    {#1 \begin{tabular}[t]{>{}<{} @{${}{}$}l}}
    {\end{tabular}\\[\belowdisplayskip]}

\lstset{%
    basicstyle=\ttfamily,
    keywordstyle=\color{blue}\ttfamily,
    stringstyle=\color{red}\ttfamily,
    commentstyle=\color{green}\ttfamily,
    morecomment=[l][\color{magenta}]{\#},
    showstringspaces=false,
}

\DeclareTextFontCommand{\emph}{\boldmath\bfseries}

\begin{document}

% \input{main-concordance}

\begin{titlepage}
    \begin{center}
        \setlength{\parindent}{0cm}
        \fontsize{16pt}{16pt}\selectfont

        \textbf{ИНСТИТУТ ТРАНСПОРТА И СВЯЗИ}

        \rule{\textwidth}{1pt}

        \vspace*{2.0cm}

        \includegraphics[width=5cm]{tsi_logo}

        \vspace*{2.0cm}

        ФАКУЛЬТЕТ КОМПЬЮТЕРНЫХ НАУК И ТЕЛЕКОММУНИКАЦИЙ

        \vspace*{2.0cm}

        Лабораторная работа №1 \\
        По дисциплине \\
        <<Численные методы и прикладное программирование>>

        \vspace*{2.0cm}
        \fontsize{14pt}{14pt}\selectfont
        Тема: \\
        <<Методы решения системы линейных уравнений. \\ Число обусловленности матрицы>>

        \vfill
        \fontsize{12pt}{12pt}\selectfont

        \begin{flushright}
            Работу выполнили: \\
            \vspace*{0.25cm}
            {\itshape Дзенис Ричард} \\
            {\itshape Кобелев Денис} \\
            {\itshape Якушин Владислав} \\
        \end{flushright}

        \vfill
        
        Рига \\ 2017 г.
    \end{center}
\end{titlepage}

\section{Формулировка задания}
\begin{itemize}
    \item Реализовать программным путём метод исключения Гаусса и итерационный метод Гаусса-Зейделя. 
    \item Результат работы программы проверить с помощью предоставленных примеров.
    \item Ручным или программным путём рассчитать число обусловленности матриц для примеров \eqref{slau:3} и \eqref{slau:5}.
    \item Для расчёта обусловленности выбрать Манхэттенскую или Евклидову норму.
    \item Составить отчёт с результатами вычислений и выводами, содержащими сравнение двух реализованных методов, а так же объяснить значения полученный при вычислении числа обусловленности матриц.
\end{itemize}

\subsection{Примеры}

\begin{equation}
    \begin{cases*}
           x_1  -2 x_2 + x_3 = 2 \\
         2 x_1  -5 x_2 - x_3 = -1 \\
        -7 x_1  +  x_3       = -2
    \end{cases*} \label{slau:1}
\end{equation}

\begin{equation}
    \begin{cases*}
        2 x_1 -   x_2 -   x_3 = 5 \\
          x_1 + 3 x_2 - 2 x_3 = 7 \\
          x_1 + 2 x_2 + 3 x_3 = 10
    \end{cases*} \label{slau:2}
\end{equation}

\begin{equation}
    \begin{cases}
         5 x_1 -5 x_2 -3 x_3 +4 = -11 \\
           x_1 -4 x_2 +6 x_3 -4 = -10 \\
        -2 x_1 -5 x_2 +4 x_3 -5 = -12 \\
        -3 x_1 -3 x_2 +5 x_3 -5 = 8
    \end{cases} \label{slau:3}
\end{equation}

\begin{equation}
    \begin{cases}
         8 x_1 + 5 x_2  + 3  x_3 = 30 \\
        -2 x_1 + 8 x_2  +    x_3 = 15 \\
           x_1 + 3 x_2  - 10 x_3 = 42
    \end{cases} \label{slau:4}
\end{equation}

\begin{equation}
    \begin{cases}
        0.78  x_1 + 0.563 x_2 = 0.217 \\
        0.913 x_1 + 0.659 x_2 = 0.254
    \end{cases} \label{slau:5}
\end{equation}

\break

\section{Метод исключения Гаусса с ведущим элементом}

\subsection{Листинг}
\lstinputlisting[language=c++, firstline=5, lastline=49]{../gauss.cpp}

\subsection{Результаты работы алгоритма}

\subsubsection{Результат работы алгоритма на примере \eqref{slau:1}}
\bash[stdout]
../solve_problem gauss 1
\END

\subsubsection{Результат работы алгоритма на примере \eqref{slau:2}}
\bash[stdout]
../solve_problem gauss 2
\END

\subsubsection{Результат работы алгоритма на примере \eqref{slau:5}}
\bash[stdout]
../solve_problem gauss 5
\END

\section{Метод Гаусса-Зейделя}

\subsection{Листинг}
\lstinputlisting[language=c++, firstline=5, lastline=27]{../gauss-seidel.cpp}

\subsection{Результаты работы алгоритма с точностью $10^{-3}$}

\subsubsection{Результат работы алгоритма на примере \eqref{slau:1}}
\bash[stdout]
../solve_problem gauss-seidel 1 0.001
\END
\newline

\subsubsection{Результат работы алгоритма на примере \eqref{slau:3}}
\bash[stdout]
../solve_problem gauss-seidel 3 0.001
\END

\subsubsection{Результат работы алгоритма на примере \eqref{slau:4}}
\bash[stdout]
../solve_problem gauss-seidel 4 0.001
\END

\section{Экспериментальное определение числа обусловленности матрицы}

\subsection{Часть листинга для нахождения числа обусловленности}
\lstinputlisting[language=python, firstline=35, lastline=48]{../find_cond.py}

\subsection{Полученные результаты}

\subsubsection{Полученные результаты для метода Гаусса}

\paragraph{Пример \eqref{slau:1}}
\bash[stdout]
../find_cond.py 1 gauss
\END

\paragraph{Пример \eqref{slau:2}}
\bash[stdout]
../find_cond.py 2 gauss
\END

\paragraph{Пример \eqref{slau:5}}
\bash[stdout]
../find_cond.py 5 gauss
\END

\subsubsection{Полученные результаты для метода Гаусса-Зейделя}

%\paragraph{Пример \eqref{slau:1}}
%\bash[stdout]
%../find_cond.py 1 gauss-seidel
%\END

\paragraph{Пример \eqref{slau:3}}
\bash[stdout]
../find_cond.py 3 gauss-seidel
\END

\paragraph{Пример \eqref{slau:4}}
\bash[stdout]
../find_cond.py 4 gauss-seidel
\END

\section{Выводы}



\end{document}

