\documentclass[a4paper]{article}
\usepackage[autostyle]{csquotes}
\usepackage{fix-cm}
\usepackage[T2A]{fontenc}
\usepackage[utf8]{inputenc}
\usepackage[russian]{babel}
\usepackage{graphicx}
\usepackage[export]{adjustbox}
\usepackage{datetime}
\usepackage{mathtools}
\usepackage{marginnote}
\usepackage{geometry}
\usepackage{titlesec}
\usepackage{nicefrac}
\usepackage{bm}
\usepackage{listings}
\usepackage{bashful}
\usepackage{pgffor}
\usepackage{amsmath}
\usepackage{amsthm}
\usepackage{amsfonts}
\usepackage{enumitem}
\usepackage{svg}
\usepackage{color}
\usepackage{xcolor}
\usepackage{colortbl}
\usepackage{tabularx}
\usepackage{multirow}
\usepackage{multicol}
\usepackage{makecell}
\usepackage{booktabs}
\usepackage{array}
\usepackage[normalem]{ulem}
\usepackage{environ}
\usepackage[singlelinecheck=false,justification=raggedleft]{caption}
% \usepackage[noae]{Sweave}

\geometry{
    a4paper,
    top=20mm,
    right=15mm,
    bottom=20mm,
    left=30mm,
}

\setcounter{secnumdepth}{3}

\titleformat{\paragraph}{\normalfont\normalsize\bfseries}{\theparagraph}{1em}{}
% \titlespacing*{\paragraph}{0pt}{3.25ex plus 1ex minus .2ex}{1.5ex plus .2ex}

% \delimitershortfall-1sp

\setlist[itemize]{noitemsep, topsep=1mm}
\setlist[enumerate]{nolistsep}

\newcommand{\includeimage}[3]{
    \begin{center}
        \includegraphics[max size={\textwidth}{100mm}]{#1}
        \captionof{figure}{#3}
        \label{#2}
    \end{center}
}

\newcommand\abs[1]{\left|#1\right|}
\newcommand\norm[1]{\left|\left|#1\right|\right|}
\newcommand\cond[1]{\text{cond}\left(#1\right)}
\newcommand{\m}[1]{\ensuremath{\bm{#1}}}
\newcommand\eqdef{\mathrel{\stackrel{\makebox[0pt]{\mbox{\normalfont\tiny def}}}{=}}}

\newcommand\numberthis{\addtocounter{equation}{1}\tag{\theequation}}

\newenvironment{where}[1][где:]
    {#1 \begin{tabular}[t]{>{}<{} @{${}{}$}l}}
    {\end{tabular}\\[\belowdisplayskip]}

\newcolumntype{\l}{@{}l@{}}
\newcolumntype{\c}{@{}c@{}}
\newcolumntype{\r}{@{}r@{}}

\newcolumntype{L}{>{$}l<{$}}
\newcolumntype{C}{>{$}c<{$}}
\newcolumntype{R}{>{$}r<{$}}

\newcolumntype{\L}{@{}L@{}}
\newcolumntype{\C}{@{}C@{}}
\newcolumntype{\R}{@{}R@{}}

\newcolumntype{\ML}{>{$\displaystyle}l<{$}}
\newcolumntype{\MC}{>{$\displaystyle}c<{$}}
\newcolumntype{\MR}{>{$\displaystyle}r<{$}}

\newcolumntype{\NL}{@{}\ML @{}}
\newcolumntype{\NC}{@{}\MC @{}}
\newcolumntype{\NR}{@{}\MR @{}}

\lstset{%
    basicstyle=\ttfamily,
    keywordstyle=\color{blue}\ttfamily,
    stringstyle=\color{red}\ttfamily,
    commentstyle=\color{green}\ttfamily,
    morecomment=[l][\color{magenta}]{\#},
    showstringspaces=false,
}

\DeclareTextFontCommand{\emph}{\boldmath\bfseries}

\begin{document}

% \input{main-concordance}
\begin{titlepage}
    \begin{center}
        \setlength{\parindent}{0cm}
        \fontsize{16pt}{16pt}\selectfont

        \textbf{ИНСТИТУТ ТРАНСПОРТА И СВЯЗИ}

        \rule{\textwidth}{1pt}

        \vspace*{2.0cm}

        \includegraphics[width=5cm]{../../tsi_logo}

        \vspace*{2.0cm}

        ФАКУЛЬТЕТ КОМПЬЮТЕРНЫХ НАУК И ТЕЛЕКОММУНИКАЦИЙ

        \vspace*{2.0cm}

        Лабораторная работа №3 \\
        По дисциплине \\
        <<Численные методы и прикладное программирование>>

        \vspace*{2.0cm}
        \fontsize{14pt}{14pt}\selectfont
        Тема: \\
        <<Методы решения нелинейного уравнения>>

        \vfill
        \fontsize{12pt}{12pt}\selectfont

        \begin{flushright}
            Работу выполнили: \\
            \vspace*{0.25cm}
            {\itshape Дзенис Ричард} \\
            {\itshape Кобелев Денис} \\
            {\itshape Якушин Владислав} \\
        \end{flushright}

        \vfill

        Рига \\ 2017 г.
    \end{center}
\end{titlepage}

\tableofcontents

\pagebreak

\section{Формулировка задания}
В данной лабораторной работе требуется реализовать два алгоритма решения нелинейного уравнения: метод бисекции и индивидуальный метод.

\section{Выводы}

\pagebreak

\section*{Приложение}

\subsection*{Вывод коэффициентов для метода парабол}

\begin{align*}
    L_2(x) = y_0l_0(x) + y_1l_1(x) + y_2l_2(x) = 0 \\
    \\
    y_0 \cdot \frac{(x - x_1)(x - x_2)}{(x_0 - x_1)(x_0 - x_2)} \\
    + y_1 \cdot \frac{(x - x_0)(x - x_2)}{(x_1 - x_0)(x_1 - x_2)} \\
    + y_2 \cdot \frac{(x - x_0)(x - x_1)}{(x_2 - x_0)(x_2 - x_1)} \\
    = 0 \\
    \\
    y_0 \cdot \frac{(x - x_1)(x - x_2)}{(x_0 - x_1)(x_0 - x_2)} \cdot \frac{(x_1 - x_0)(x_1 - x_2)(x_2 - x_0)(x_2 - x_1)}{(x_1 - x_0)(x_1 - x_2)(x_2 - x_0)(x_2 - x_1)} \\
    + y_1 \cdot \frac{(x - x_0)(x - x_2)}{(x_1 - x_0)(x_1 - x_2)} \cdot \frac{(x_0 - x_1)(x_0 - x_2)(x_2 - x_0)(x_2 - x_1)}{(x_0 - x_1)(x_0 - x_2)(x_2 - x_0)(x_2 - x_1)} \\
    + y_2 \cdot \frac{(x - x_0)(x - x_1)}{(x_2 - x_0)(x_2 - x_1)} \cdot \frac{(x_0 - x_1)(x_0 - x_2)(x_1 - x_0)(x_1 - x_2)}{(x_0 - x_1)(x_0 - x_2)(x_1 - x_0)(x_1 - x_2)} \\
    = 0 \\
    \\
    y_0 \cdot (x - x_1)(x - x_2) \cdot (x_1 - x_0)(x_1 - x_2)(x_2 - x_0)(x_2 - x_1) \\
    + y_1 \cdot (x - x_0)(x - x_2) \cdot (x_0 - x_1)(x_0 - x_2)(x_2 - x_0)(x_2 - x_1) \\
    + y_2 \cdot (x - x_0)(x - x_1) \cdot (x_0 - x_1)(x_0 - x_2)(x_1 - x_0)(x_1 - x_2) \\
    = 0 \\
    \\
    - y_0 \cdot (x - x_1)(x - x_2) \cdot (x_1 - x_0)(x_1 - x_2)^2(x_2 - x_0) \\
    - y_1 \cdot (x - x_0)(x - x_2) \cdot (x_0 - x_1)(x_0 - x_2)^2(x_2 - x_1) \\
    - y_2 \cdot (x - x_0)(x - x_1) \cdot (x_0 - x_1)^2(x_0 - x_2)(x_1 - x_2) \\
    = 0 \\
    \\
    - y_0 \cdot (x - x_1)(x - x_2) \cdot (x_0 - x_1)(x_1 - x_2)^2(x_0 - x_2) \\
    + y_1 \cdot (x - x_0)(x - x_2) \cdot (x_0 - x_1)(x_0 - x_2)^2(x_1 - x_2) \\
    - y_2 \cdot (x - x_0)(x - x_1) \cdot (x_0 - x_1)^2(x_0 - x_2)(x_1 - x_2) \\
    = 0 \\
    \\
    (x_0 - x_1)(x_1 - x_2)(x_0 - x_2) \cdot \big( \\
        - y_0 \cdot (x - x_1)(x - x_2) \cdot (x_1 - x_2) \\
        + y_1 \cdot (x - x_0)(x - x_2) \cdot (x_0 - x_2) \\
        - y_2 \cdot (x - x_0)(x - x_1) \cdot (x_0 - x_1) \\
    \big) = 0
\end{align*}
Введём следующие обозначения:
\begin{align*}
    m = (x_0 - x_1) \\
    n = (x_1 - x_2) \\
    k = (x_0 - x_2)
\end{align*}
Тогда:
\begin{align*}
    - y_0 \cdot (x - x_1)(x - x_2) \cdot n \\
    + y_1 \cdot (x - x_0)(x - x_2) \cdot k \\
    - y_2 \cdot (x - x_0)(x - x_1) \cdot m \\
    = 0 \\
    \\
    - y_0 \cdot (x^2 - xx_2 - xx_1 + x_1x_2) \cdot n \\
    + y_1 \cdot (x^2 - xx_2 - xx_0 + x_0x_2) \cdot k \\
    - y_2 \cdot (x^2 - xx_1 - xx_0 + x_0x_1) \cdot m \\
    = 0 \\
    \\
    - y_0 \cdot (x^2 - x(x_2 + x_1) + x_1x_2) \cdot n \\
    + y_1 \cdot (x^2 - x(x_2 + x_0) + x_0x_2) \cdot k \\
    - y_2 \cdot (x^2 - x(x_1 + x_0) + x_0x_1) \cdot m \\
    = 0
\end{align*}
Введём следующие обозначения: \\
\begin{center}
    \begin{tabular}{CC}
        \begin{matrix}
            e = x_2 + x_1 \\
            f = x_2 + x_0 \\
            g = x_1 + x_0
        \end{matrix}
        &
        \begin{matrix}
            o = x_1x_2 \\
            p = x_0x_2 \\
            q = x_0x_1
        \end{matrix}
    \end{tabular}
\end{center}
Тогда:
\begin{align*}
    - y_0 \cdot (x^2 - xe + o) \cdot n \\
    + y_1 \cdot (x^2 - xf + p) \cdot k \\
    - y_2 \cdot (x^2 - xg + q) \cdot m \\
    = 0 \\
    \\
    (-y_0n + y_1k - y_2m)x^2 \\
    - (-y_0ne + y_1kf - y_2mg)x \\
    + (-y_0no + y_1kp - y_2mq) \\
    = 0 \\
\end{align*}
Введём следующие обозначения:
\begin{align*}
    u = y_0n \\
    v = y_1k \\
    w = y_2m
\end{align*}
\begin{align*}
    a = -u + v - w \\
    b = -ue + vf - wg \\
    c = -uo + vp - wq \\
\end{align*}

Далее решается квадратное уравнение, при условии $a \neq 0$.
\begin{align*}
    ax^2 - bx + c = 0
\end{align*}
\begin{align*}
    x_{1,2} = \frac{b \pm \sqrt{b^2 - 4ac}}{2a}
\end{align*}

В случае $a = 0$, уравнение -- линейное, и принимает вид:
\begin{align*}
    -bx + c = 0 \\
    x = \frac{c}{b}
\end{align*}

\subsection*{Вывод коэффициентов для метода парабол (версия 2)}

\begin{align*}
    L_2(x) = y_0l_0(x) + y_1l_1(x) + y_2l_2(x) = 0 \\
    \\
    y_0 \cdot \frac{(x - x_1)(x - x_2)}{(x_0 - x_1)(x_0 - x_2)} \\
    + y_1 \cdot \frac{(x - x_0)(x - x_2)}{(x_1 - x_0)(x_1 - x_2)} \\
    + y_2 \cdot \frac{(x - x_0)(x - x_1)}{(x_2 - x_0)(x_2 - x_1)} \\
    = 0
\end{align*}

Каждую из дробей с коэффициентом $y_i$ можно представить как квадратное уравнение:
\begin{align*}
    y_0 \cdot \frac{(x - x_1)(x - x_2)}{(x_0 - x_1)(x_0 - x_2)} = y_0 \cdot \frac{x^2 - x(x_1 + x_2) + x_1x_2}{(x_0 - x_1)(x_0 - x_2)} = a_0x^2 + b_0x + c_0, \text{где}
    \begin{cases}
        a_0 = \displaystyle\frac{y_0}{(x_0 - x_1)(x_0 - x_2)} \\
        \\
        b_0 = \displaystyle\frac{y_0(x_1 + x_2)}{(x_0 - x_1)(x_0 - x_2)} \\
        \\
        c_0 = \displaystyle\frac{y_0x_1x_2}{(x_0 - x_1)(x_0 - x_2)}
    \end{cases}
\end{align*}

Аналогично можно проделать для дробей при коэффициентах $y_1$ и $y_2$.
Тогда исходное уравнение с тремя дробями можно представить в виде следующего квадратного уравнения:
\begin{equation*}
    Ax^2 + Bx + C = 0
\end{equation*}
где:
\begin{align*}
    A &= a_0 + a_1 + a_2 \\
    B &= b_0 + b_1 + b_2 \\
    C &= c_0 + c_1 + c_2
\end{align*}

\end{document}

